\section{Key-Value Stores as a Concurrency Control Testbed} \label{sec:problem}

In our tests, we wished to explore the performance effects of HTM in a simple
environment, with relatively few confounding factors that could impact
experimental results. We therefore chose to implement concurrency control for an
\textit{in-memory key-value store}.

A key-value store presents an associative data access interface: it maps a set
of keys $K$ to a set of values $V$, where the keys and values can be arbitrary
objects. The keys are generally prehashed to obtain an uniform distribution. The
values can be stored separately, with fixed-length pointers representing them in
the key-value store implementation. The primary operations provided by a
key-value store's interface are \textsc{GET}(key), \textsc{PUT}(key,value), and
\textsc{DELETE}(key).

Despite their simplicity, in-memory key-value stores present many of the same
concurrency control challenges as more complex database systems. Readers can
conflict with writers writing the same key, and writers can conflict with other
writers. Thus, some concurrency control mechanism is required to provide
isolation between conflicting accesses. Furthermore, clients of a key-value
store often perform operations on multiple keys in a single transaction (a
\textit{multi-key transaction}). Clients generally require that such a
transaction appear atomic and isolated: it must commit all changes or none, and
it should not see effects from other incomplete transactions. Providing such
synchronization for multi-key transactions is the focus of this project.

There are several possible levels of isolation a concurrency control mechanism
could provide. At a minimum, most datastore clients require \textit{read
  committed} isolation -- all values read have been committed by some
transaction. Multi-key transactions also raise the possibility that a
transaction could read a value and write another value based on it, unaware that
the value has been updated in the interim. Many datastore systems therefore also
guarantee \textit{repeatable reads} -- if a value were to be read multiple times
within a transaction, the same value would be read each time. Because it is so
frequently required and raises additional concurrency control challenges, this
is the level of isolation we implemented in each of the concurrency control
schemes we tested.
