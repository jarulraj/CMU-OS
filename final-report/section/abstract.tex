
\begin{abstract}
  Hardware transactional memory (HTM) has long been theorized to offer many
  benefits for concurrent programming, but a dearth of implementations has
  inhibited real-world tests. Recent Intel processors have introduced support
  for HTM in the form of transaction synchronization instruction set
  extensions. These extensions provide two software interfaces to define
  transaction regions: a flexible interface which allows the programmer to
  specify a fallback path to handle transaction failures (Restricted
  Transactional Memory) and another backward-compatible interface that is less
  customisable (Hardware Lock Elision). We applied these new hardware primitives
  to control concurrency in transactions on an in-memory key-value store, and
  compared their performance against traditional pessimistic concurrency control
  schemes.  We tested multiple-key transactions with both static and dynamic
  read/write sets. Our results indicate that, at least in this setting, HTM
  performance is essentially comparable with similarly structured lock-based
  schemes. HTM thus appears unlikely to offer many performance benefits, though
  it may still ease the burden on programmers.
\end{abstract}
