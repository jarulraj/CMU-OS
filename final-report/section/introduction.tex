\section{Introduction} \label{sec:intro}

Most conventional concurrency techniques are pessimistic -- they rely on some
form of mutual exclusion to ensure isolation between concurrent
transactions. Using locks, however, can impose significant overhead. Under some
circumstances, better performance can be obtained with optimistic concurrency
control (OCC), in which transactions execute speculatively, and are then
verified before being allowed to commit.

One model for OCC is transactional memory (TM). Under the TM model, the
programmer is presented with an abstraction in which a section of code can be
marked as a transaction. All memory reads and writes within the transaction are
optimistically executed, and some underlying layer of execution guarantees that
the transaction is only allowed to commit if transactional semantics can be
maintained. This allows read/write access to multiple memory words, while still
maintaining isolation in a lock-free manner \citep{Herlihy93}.

TM can be realized either in hardware or in software. For instance, the
load-linked/store-conditional primitive available in modern processors is a
special case of hardware TM, where access is regulated for only a single memory
word. Hardware implementations generally rely on extensions to the
cache-coherence protocol, whereas software implementations provide transactional
interfaces via basic hardware primitives like atomic compare-and-swap
(CAS). These software-based approaches tend to introduce too much overhead to be
competitive with lock-based solutions. Hardware approaches, on the other hand,
can be implemented efficiently, making them very appealing in principle. Indeed,
simulations have suggested that hardware transactional memory (HTM) can match or
outperform pessimistic schemes. However, very few practical chips to date have
supported HTM, which has led to a dearth of actual experiments confirming the
utility of the method. In particular, there have been few published results on
the impact of HTM on datastore systems, one of the primary motivations for
transactional semantics.

Recent Intel Haswell processors have brought transactional memory support to
non-mainframe systems for the first time. The Haswell processors implement HTM
in the form of instruction set extensions known as Transaction Synchronization
Extensions (TSX). TSX supports two interfaces \citep{tsx-intro}: \\

\begin{itemize} 
\item \textbf{Hardware Lock Elision (HLE):} \\ This is a backwards-compatible
  extension which allows specifying transaction regions via \textrm{XACQUIRE}
  and \textrm{XRELEASE} instruction prefixes. HLE is compatible with the
  lock-based programming model. In HLE, when the transaction execution fails,
  the TM implementation simply reverts to normal lock semantics during
  re-execution. \\
\item \textbf{Restricted Transactional Memory (RTM):} \\ This is a more flexible
  extension, albeit not a backward compatible one, which allows specification of
  transactions using \textrm{XBEGIN, XEND and XABORT} instructions. It allows
  the programmer to specify the fallback path for transaction failures. \\
\end{itemize}

In this paper, we contribute a study of the effects of these Haswell HTM
implementations on synchronous access to in-memory data
structures. Specifically, we focus on \textit{multi-key transactions} on a
\textit{key-value store}, comparing the performance of an HTM concurrency
control with pessimistic alternatives. We also study the performance of HTM in
the case of \textit{dynamic read/write sets}, which historically lead to much
worse performance in pessimistic schemes.

In \Cref{sec:tm}, we describe the problem of synchronizing multi-key
transactions on a key-value store using hardware TM support, and related work to
date on this problem. We then present the traditional pessimistic concurrency
control protocols we use for comparison in \Cref{sec:pessimistic}. In
\Cref{sec:eval}, we describe the test framework we used, the tests we performed,
and the results we obtained. Finally, we summarize the conclusions we drew from
our results in \Cref{sec:conclusion}.
