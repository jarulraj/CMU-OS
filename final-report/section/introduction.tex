\section{Introduction} \label{sec:intro}


Concurrent programming is challenging due to larger state space and our natural  
sequential perspective. Hardware transactional memory (HTM) can  
simplify concurrent programming by supporting memory transactions \citep{tm}. 
To better understand the practical benefits of HTM, we implemented a main-memory 
key-value store that supports multi-key transactions by leveraging the 
primitives provided by HTM. 

We motivate the problem addressed by this project along with related work in
\Cref{sec:related}.
In \Cref{sec:problem}, we describe the problem of synchronizing multi-key
transactions on a key-value store. In \Cref{sec:tm}, we outline the operation of
HTM, its use in transaction management, and related work to date on TM and HTM
performance. We then present the traditional pessimistic concurrency control
protocols, against which we compare HTM, in \Cref{sec:pessimistic} . In
\Cref{sec:eval}, we describe the test framework we used, the tests we performed,
and the results we obtained. Finally, we summarize the conclusions we draw from
our results in \Cref{sec:conclusion}.

\section{Motivation and Related Work} \label{sec:related}

Most conventional concurrency techniques are pessimistic -- they rely on some
form of mutual exclusion to ensure isolation between concurrent
transactions. Using locks, however, can impose significant overhead. Under some
circumstances, better performance can be obtained with optimistic concurrency
control (OCC), in which transactions execute speculatively, and are then
verified before being allowed to commit.

One model for OCC is transactional memory (TM). Under the TM model, the
programmer is presented with an abstraction in which a section of code can be
marked as a transaction. All memory reads and writes within the transaction are
optimistically executed, and some underlying layer of execution guarantees that
the transaction is only allowed to commit if transactional semantics can be
maintained. This allows read/write access to multiple memory words, while still
maintaining isolation in a lock-free manner \citep{Herlihy93}.

TM can be realized either in hardware or in software. For instance, the
load-linked/store-conditional primitive available in modern processors is a
special case of hardware TM, where access is regulated for only a single memory
word. Hardware implementations generally rely on extensions to the
cache-coherence protocol \citep{htm}, whereas software implementations provide transactional
interfaces via basic hardware primitives like atomic compare-and-swap
(CAS). These software-based approaches tend to introduce too much overhead to be
competitive with lock-based solutions. Algorithms like K-means and 
B+ tree indexing have been observed to perform $4-100$X
slower than the serial non-transactionalized version in different 
software TM implementations \citep{cacm08}. 
Hardware approaches, on the other hand,
can be implemented efficiently, making them very appealing in principle. Indeed,
simulations have suggested that hardware transactional memory (HTM) can match or
outperform pessimistic schemes. However, very few practical chips to date have
supported HTM, which has led to a dearth of actual experiments confirming the
utility of the method. In particular, there have been few published results on
the impact of HTM on datastore systems, one of the primary motivations for
transactional semantics.

Recent Intel Haswell processors have brought transactional memory support to
non-mainframe systems with potential for a widespread impact \citep{tsx-tools,lwn}. 
In this paper, we contribute a study
of the effects of these Haswell HTM implementations on concurrency control over 
accesses to
in-memory data structures. Specifically, we focus on \textit{multi-key
  transactions} on a \textit{key-value store}, comparing the performance of
HTM-based concurrency control with pessimistic alternatives. We also study the
performance of HTM while handling \textit{dynamic read/write sets}, which
generally leads to liveness problems in pessimistic schemes.

