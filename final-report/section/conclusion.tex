\section{Conclusions} \label{sec:conclusion}

There are a number of interesting conclusions that can be drawn from our
results. Most importantly, our experiments suggest that hardware transactional
memory is unlikely to provide any significent performance improvement over
traditional, software-based concurrency schemes in most situations.

Additionally, we did not feel that at this time it provides a large engineering
benefit in making concurrent code simpler or easier to get right. This is
primarily due to the fact that hardware transaction memory is still in
development and is not fully documented, whereas there exist many tools for
implementing the traditional software based methods. (In particular, we found
the C++11 concurrency libraries effective and easy to use.)

However, hardware transactional memory is certainly a perfectly viable option
for implementing concurrency control in a datastore, including support for
multi-key transactions and dynamic read/write sets. It is already easier to
handle dynamic read/write sets in HTM than in conventional schemes, and we do
see the potential for more engineering benefits in the future, as the technology
becomes more mature and well-documented.
